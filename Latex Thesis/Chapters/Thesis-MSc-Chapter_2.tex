% #############################################################################
% This is Chapter 2
% !TEX root = ../main.tex
% #############################################################################
% Change the Name of the Chapter i the following line
\fancychapter{Background}
\cleardoublepage
% The following line allows to ref this chapter
\label{chap:2background}

The journey from the discovery of animal electricity to the utilization of voltaic alternate electricity in our homes took centuries of research and experiments. As society expanded in both population and complexity, the entire power system evolved into the advanced infrastructure we rely on today\cite{rahman_basicsElectricity_2016}.

According to Bollen and Hassan (2011)\cite{integrationdg_bollen2011}, a power system is composed by three elements: electricity production units, electrical devices that consumes from it, and a power grid that connects them.

Introducing new energy resources presents both advantages and disadvantages. On the positive side, energy generation can be economically beneficial to society, leading to an increase in available energy. When it comes to renewable energy resources, there is also the environmental benefit, as their use helps mitigate the dependence on fossil-fueled based production that is a primary contributor to \acp{GHG} emissions, which is widely recognized as a key driver of global warming.

Renewable energy resources are defined by \cite{renewableenergies_rybar2015} as energy that is derived from natural processes and renewed within a human lifespan. these sources can be derived from the sun (photovoltaic or heat), wind, geothermal, water flow or tides. Unlike fossil-fueled energy sources, which are finite, renewable energy is, in a certain way, unlimited as it constantly replenishes by natural phenomena. An axiomatic characteristic of renewable energy is its sustainability, making it an essential factor to reduce \acp{GHG} and mitigate climate change. Another axiom is that these sources renew themselves over short periods, ensuring a constant supply without draining Earth's resources.

However, this also raises concerns regarding power quality and the reliability challenges posed by the integration of different energy sources into the grid, along with several other factors that will be discussed  later in this chapter.

% #############################################################################

\section{Integration Challenges}

As power systems shift from large conventional production units that usually generates energy from fossil fuels, to more decentralized renewable electricity production, several challenges arise due to the irregular and dispersed nature of these sources.

A reliable energy source is steady, abundant, and has a predictable output. But when it comes to \acp{DER}, although many are abundant, they have very low predictability of production. For example, solar energy, while emission-free during operation, introduces complexity in its integration due to its intermittent nature and dependence on daylight. Some factors that are out of human control such as weather conditions, time of day, seasonal and regional variability impact the generation output. Moreover, in order to produce in large-scale it requires extensive land and high initial investment.

% Similarly, wind power

% Similarly, wind energy contributes significant potential to power systems but presents integration hurdles tied to the inconsistency of wind. Regions with stable wind patterns can benefit from efficient energy generation, yet the unpredictable nature of wind can lead to imbalances in supply. This intermittency necessitates the development of advanced forecasting tools and flexible grid responses to avoid disruptions. Additionally, wind turbines occupy considerable land or offshore space, leading to further complications in grid connectivity and infrastructure development, particularly in remote areas.

% Hydroelectric power, while offering a reliable and consistent energy source, poses its own integration challenges. Despite its ability to ramp up or down quickly to match demand, large hydroelectric projects can cause significant ecological disruptions and may be limited by regional water availability, especially under changing climate conditions. Grid operators must consider both the environmental impacts and the dependency on water resources when evaluating the hosting capacity of hydroelectric systems.

% Biomass energy, though versatile and capable of utilizing various organic materials for power generation, faces challenges related to emissions and sustainability. While cleaner than fossil fuels, biomass combustion still releases pollutants, and the sustainability of its feedstocks can impact the overall environmental benefits. From a grid perspective, ensuring consistent biomass availability and maintaining efficient integration require careful management of resource supply chains.

% Finally, geothermal energy, known for its reliability and minimal land footprint, offers a stable power supply, making it an appealing option for grid stability. However, its application is geographically constrained, limiting the expansion of geothermal systems to areas with significant tectonic activity. High upfront costs for infrastructure, particularly in deep drilling, add to the economic challenges of increasing geothermal hosting capacity in power systems.

% As these renewable energy sources become more prevalent, their integration into existing power systems highlights the need for technological advancements in grid management, real-time monitoring, and storage solutions. Each source introduces unique requirements and constraints, necessitating a coordinated approach to grid development that enhances the capacity to host and balance diverse renewable resources. Understanding the specific demands of these energy types is essential for overcoming integration challenges and ensuring a stable and efficient energy transition.


% #############################################################################

\section{Hosting Capacity in Power Systems}

\subsection{Hosting Capacity limiting factors}
\cite{IEEEDataPort}

\subsection{Overview of HC calculation methods}




% #############################################################################

\section{Spatial Mapping in HC}



% #############################################################################

\section{Gaps and future challenges}



