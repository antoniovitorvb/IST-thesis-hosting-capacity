% #############################################################################
% This is Chapter 1
% !TEX root = ../main.tex
% #############################################################################
% Change the Name of the Chapter i the following line
\fancychapter{Introduction}
\cleardoublepage
% The following line allows to ref this chapter
\label{chap:1intro}

\section{Context and Motivation}


% #############################################################################

\section{Problem Statement}

One of the main problems within today’s electrical energy is the growing discrepancy between the energy demand and generation. Despite the technological advancements in energy generation, new energy intense public needs impose challenges, such as the increase of \ac{EV}\cite{iea2022}, economic growth and improved living standards that led to higher average energy demand per capita\cite{owid-energy-access, ember2024electricity} and processes such training AI models.

According to the International Energy Agency (IEA), \ac{EV} sales nearly doubled to 6.6 million units in 2021. These sales alone accounted for 9\% of the global car market\cite{iea2022}. Between 2002 and 2022, the global population increased by 26.6\%, from approximately 6.3 billion to 8 billion\cite{un_population_2024}. Over the same period, the global average annual electricity consumption per capita rose by 11.3\%, from 22,758.68 MWh to 25,331.72 MWh\cite{owid-energy-access, ember2024electricity}. This data highlights the rise in per capita electricity demand alongside significant population growth.

This imbalance raises concerns regarding sustainable development highlighting the critical importance of energy generation capacities. Numerous studies have identified significant challenges associated with increasing energy generation, including the intermittency of renewable energy sources, infrastructural limitations, energy losses, and limited information within \ac{LV} grids\cite{abideen_ReviewToolsMethods, pcarvalho2024lecture}.



% #############################################################################

\section{Research Objectives}



% #############################################################################

\section{Research Questions}



% #############################################################################

\section{Scope and Delimitation}



% #############################################################################

\section{Organization of the Document}

This thesis is is organized as follows: 
\begin{inparaenum}[(a)]
\item \Cref{chap:1intro};
\item \cref{chap:2background};
\item \cref{chap:3methodology};
\item \Cref{chap:4analysis};
\item \cref{chap:5discussion}; and
\item \Cref{chap:6conclusion}
\end{inparaenum}
